\documentclass[12pt]{article} 


\usepackage[utf8]{inputenc}    
\usepackage[T1]{fontenc}       
\usepackage{lmodern}    
\usepackage{array}       
\usepackage{amsmath, amssymb}  
\usepackage{graphicx}          
\usepackage{geometry}          
\usepackage{hyperref}          
\usepackage{fancyhdr}          
\usepackage{parskip}           
\usepackage{tikz}              
\usepackage{booktabs}          
\usepackage{enumitem}          
\usepackage{caption}           
\usepackage{listings}          
\usepackage{multirow}          
\usepackage{amsthm}
\usepackage{longtable}
\lstset{
  frame=tb,
  language=C,
  aboveskip=3mm,
  belowskip=3mm,
  showstringspaces=false,
  columns=flexible,
  basicstyle={\small\ttfamily},
  numbers=none,
  numberstyle=\tiny\color{gray},
  keywordstyle=\color{blue},
  commentstyle=\color{brown},
  stringstyle=\color{orange},
  breaklines=true,
  breakatwhitespace=true,
  tabsize=3
}


\geometry{margin=1in} 


\pagestyle{fancy}
\fancyhf{}
\fancyhead[L]{\leftmark} 
\fancyhead[R]{\thepage}  

\title{MIPS}
\author{WANG Xiyu}
\date{\today}
\theoremstyle{definition}
\newtheorem*{remark}{Remarks}
\begin{document}

\maketitle

\tableofcontents 

\section{Instruction Set Architecture}
Instruction Set Architecture includes everything programers need to know to make the machine code work correctly, and allows computer designers to talk about functions independently from the hardware that performs them. 
This abstraction allows many implementations of varyin cost and performance that can run identical software. Basically an API between software and hardware.
\section{Machine Code vs Assembly Language}
\section{Walkthrough}
The major 2 components of a computer are the processor and memory(RAM), which are connected by bus, which is a data connector.
The code and data reside in memory, which are transferred into the processor duting execution. 
The Processor is much faster than memory access, therefore we have load-store model that limits the memory operations and relies on registers for storage during execution.
\subsection{Memory Instruction}
To move data from memory to registers, and the other way round later.
\subsubsection*{Variable mapping}
mapping of variables to registers.
\begin{lstlisting}
    r0 <- load i
\end{lstlisting}
\subsection{Reg-to-Reg Arithmetic, and other calculation instructions}
Arithmetic opeartions are be done directly on registers only, therefore very fast.
\subsection{Control flow instructions}
Instructions that changes the order of sequential execution.
\section{General Purpose Register}
There are 32 general purpose registers in a mips processor, from number 0 to number 31, each with a name and corresponding functionality.

\begin{longtable}{|p{1.5cm}|p{2cm}|p{3cm}|p{7cm}|}
    \hline
    \textbf{Reg. Number} & \textbf{Reg. Name} & \textbf{Alias} & \textbf{Remarks} \\ 
    \hline
    0  & \$zero  & constant zero & Always holds the constant value 0. Cannot be modified. \\ 
    \hline
    1  & \$at    & assembler temporary & Reserved for use by the assembler. Not typically used in normal code. \\ 
    \hline
    2-3  & \$v0-\$v1  & values for results & Used to store function return values and results of expressions. \\ 
    \hline
    4-7  & \$a0-\$a3  & arguments & Used to pass the first four arguments to functions. Additional arguments are passed via the stack. \\ 
    \hline
    8-15  & \$t0-\$t7  & temporaries & Temporary registers not preserved across function calls. Free to use within functions. \\ 
    \hline
    16-23 & \$s0-\$s7  & saved temporaries & Temporary registers, but values must be preserved across function calls (callee-saved). \\ 
    \hline
    24-25 & \$t8-\$t9  & more temporaries & Additional temporary registers, not preserved across function calls. \\ 
    \hline
    26-27 & \$k0-\$k1  & kernel reserved & Reserved for the operating system kernel. Do not use in user-level code. \\ 
    \hline
    28    & \$gp       & global pointer & Points to the middle of the global/static data segment. \\ 
    \hline
    29    & \$sp       & stack pointer  & Points to the top of the current stack frame. Grows downward. \\ 
    \hline
    30    & \$fp       & frame pointer  & Used as a frame pointer for function calls. Can also be \$s8 in some conventions. \\ 
    \hline
    31    & \$ra       & return address & Stores the return address for function calls. Used by the `jal` instruction. \\ 
    \hline
    \caption{MIPS General Purpose Registers with Remarks}
    \end{longtable}
\section{MIPS Assembly Language}
\subsection{General Instruction Syntax}
There are 3 types of MIPS instruction, R type (register), I type (immediate), and J type (jump).
All instructions have 32 bits, which fit into 4 bytes blocks in the memory, or the capacity of one register.
\begin{longtable}{|>{\centering\arraybackslash}p{2cm}|>{\centering\arraybackslash}p{3cm}|>{\centering\arraybackslash}p{3cm}|>{\centering\arraybackslash}p{6cm}|}
    \hline
    \textbf{Format Type} & \textbf{Field Name} & \textbf{Field Size (bits)} & \textbf{Description} \\ 
    \hline
    \textbf{R-Type} & Opcode & 6 & The operation code, always `000000` for R-type instructions. \\ 
    \cline{2-4}
                    & rs     & 5 & The first source register. \\ 
    \cline{2-4}
                    & rt     & 5 & The second source register. \\ 
    \cline{2-4}
                    & rd     & 5 & The destination register where the result is stored. \\ 
    \cline{2-4}
                    & shamt  & 5 & The shift amount, used only for shift instructions. Typically zero for other instructions. \\ 
    \cline{2-4}
                    & funct  & 6 & Function field that determines the specific operation (e.g., add, sub). \\ 
    \hline
    \textbf{I-Type} & Opcode & 6 & The operation code identifying the instruction type. \\ 
    \cline{2-4}
                    & rs     & 5 & The source register. \\ 
    \cline{2-4}
                    & rt     & 5 & The destination register. \\ 
    \cline{2-4}
                    & Immediate & 16 & A 16-bit immediate value or address offset. \\ 
    \hline
    \textbf{J-Type} & Opcode & 6 & The operation code identifying the instruction type. \\ 
    \cline{2-4}
                    & Address & 26 & The target address for jump instructions. \\ 
    \hline
    \caption{R, I, and J Type Instruction Formats in MIPS}
    \end{longtable}

Notes: The order of registers in MIPS instruction lines are different from the order in the instruction format. \newline
In 32 bits format the register order of R type instructions is rs, rt, rd; i
\subsection{Arithmetic Operation: Addition}
\subsection{Arithmetic Operation: Subtraction}
\subsection{Complex Expressions}
\subsection{Constant/Immediate Operand}
\subsection{Register Zero (\$0 or \$zero)}
\subsection{Logical Operation: Overview}
\subsection{Logical Operation: Shifting}
\subsection{Logical Operation: Bitwise AND}
\subsection{Logical Operation: Bitwise OR} 
\subsection{Logical Operation: Bitwise NOR}
\subsection{Logical Operation: Bitwise XOR}
\section{Large Constant: Case study}
\section{MIPS Basic Instruction Checklist}


\begin{longtable}{|l|l|p{6cm}|p{6cm}|}
    \hline
    \textbf{Instruction} & \textbf{Format} & \textbf{Operation} & \textbf{Description} \\
    \hline
    \endfirsthead
    \hline
    \textbf{Instruction} & \textbf{Format} & \textbf{Operation} & \textbf{Description} \\
    \hline
    \endhead
    \hline
    \endfoot
    \hline
    \endlastfoot
    add  & R  & \raggedright $d = s + t$ & Adds two registers and stores the result in a register. \\
    \hline
    addi & I  & \raggedright $t = s + imm$ & Adds a register and an immediate value and stores the result in a register. \\
    \hline
    sub  & R  & \raggedright $d = s - t$ & Subtracts one register from another and stores the result in a register. \\
    \hline
    mult & R  & \raggedright $LO = s \times t$ & Multiplies two registers and stores the result in special registers LO (low) and HI (high). \\
    \hline
    div  & R  & \raggedright $LO = s / t$ & Divides one register by another and stores the quotient in LO and the remainder in HI. \\
    \hline
    and  & R  & \raggedright $d = s \land t$ & Performs a bitwise AND of two registers and stores the result in a register. \\
    \hline
    andi & I  & \raggedright $t = s \land imm$ & Performs a bitwise AND of a register and an immediate value. \\
    \hline
    or   & R  & \raggedright $d = s \lor t$ & Performs a bitwise OR of two registers and stores the result in a register. \\
    \hline
    ori  & I  & \raggedright $t = s \lor imm$ & Performs a bitwise OR of a register and an immediate value. \\
    \hline
    xor  & R  & \raggedright $d = s \oplus t$ & Performs a bitwise XOR of two registers and stores the result in a register. \\
    \hline
    xori & I  & \raggedright $t = s \oplus imm$ & Performs a bitwise XOR of a register and an immediate value. \\
    \hline
    nor  & R  & \raggedright $d = \lnot(s \lor t)$ & Performs a bitwise NOR (NOT OR) of two registers and stores the result in a register. \\
    \hline
    sll  & R  & \raggedright $d = t << shamt$ & Shifts a register value left by the shift amount (shamt) and stores the result. \\
    \hline
    srl  & R  & \raggedright $d = t >> shamt$ & Shifts a register value right by the shift amount (shamt) and stores the result. \\
    \hline
    sra  & R  & \raggedright $d = t >> shamt$ (arithmetic) & Shifts a register value right by the shift amount (shamt) with sign extension. \\
    \hline
    slt  & R  & \raggedright $d = (s < t)$ & Sets the destination register to 1 if the first source register is less than the second; otherwise, sets it to 0. \\
    \hline
    slti & I  & \raggedright $t = (s < imm)$ & Sets the destination register to 1 if the source register is less than the immediate value; otherwise, sets it to 0. \\
    \hline
    lw   & I  & \raggedright $t = \text{Mem}[s + \text{offset}]$ & Loads a word from memory into a register. \\
    \hline
    sw   & I  & \raggedright $\text{Mem}[s + \text{offset}] = t$ & Stores a word from a register into memory. \\
    \hline
    lb   & I  & \raggedright $t = \text{Mem}[s + \text{offset}]$ & Loads a byte from memory into a register (sign-extended). \\
    \hline
    lbu  & I  & \raggedright $t = \text{Mem}[s + \text{offset}]$ & Loads a byte from memory into a register (zero-extended). \\
    \hline
    sb   & I  & \raggedright $\text{Mem}[s + \text{offset}] = t$ & Stores a byte from a register into memory. \\
    \hline
    lh   & I  & \raggedright $t = \text{Mem}[s + \text{offset}]$ & Loads a halfword from memory into a register (sign-extended). \\
    \hline
    lhu  & I  & \raggedright $t = \text{Mem}[s + \text{offset}]$ & Loads a halfword from memory into a register (zero-extended). \\
    \hline
    sh   & I  & \raggedright $\text{Mem}[s + \text{offset}] = t$ & Stores a halfword from a register into memory. \\
    \hline
    beq  & I  & \raggedright \texttt{if} $s = t$ \texttt{then} $PC = PC + 4 + (imm \times 4)$ & Branches to a label if two registers are equal. \\
    \hline
    bne  & I  & \raggedright \texttt{if} $s \neq t$ \texttt{then} $PC = PC + 4 + (imm \times 4)$ & Branches to a label if two registers are not equal. \\
    \hline
    j    & J  & \raggedright $PC = (PC \& 0xF0000000) | (\text{address} \times 4)$ & Jumps to a specified address. \\
    \hline
    jr   & R  & \raggedright $PC = s$ & Jumps to the address contained in a register. \\
    \hline
    jal  & J  & \raggedright $RA = PC + 4; PC = (PC \& 0xF0000000) | (\text{address} \times 4)$ & Jumps to a specified address and saves the return address in the link register. \\
    \hline
    mfhi & R  & \raggedright $d = HI$ & Moves the contents of the HI register to a general-purpose register. \\
    \hline
    mflo & R  & \raggedright $d = LO$ & Moves the contents of the LO register to a general-purpose register. \\
    \hline
    mthi & R  & \raggedright $HI = s$ & Moves the contents of a general-purpose register to the HI register. \\
    \hline
    mtlo & R  & \raggedright $LO = s$ & Moves the contents of a general-purpose register to the LO register. \\
    \hline
    nop  & R  & \raggedright No operation & Does nothing; often used for delay slots. \\
    \hline
\end{longtable}
\section*{Writing MIPS instructions and MIPS to C translation}
General strategy: 
\begin{enumerate}
    \item 
\end{enumerate}


\section{The Processor - Datapath}
\subsection{Datapath and Control}
Data path consists of components that process data, which performs the arithmetic, logical and memory operations.
Control tells the datapath, memory and I/O devices what to do according to program instructions.
\subsection{MIPS Processor implementation}
Implement a subset of the core MIPS ISA
\subsubsection{Arithmatic and Logical operations}
\begin{itemize}
    \item add
    \item sub
    \item and 
    \item or 
    \item addi
    \item slt
\end{itemize}
Note: andi and ori are not supported with current processor design as we always do sign extension on immediate value.
\subsubsection{Data transfer instructions}
\begin{itemize}
    \item lw 
    \item sw 
\end{itemize}
\subsubsection{Branches}
\begin{itemize}
    \item beq
    \item bne
\end{itemize}

\subsection{Instruction Execution Cycle}
The execution on ONE instruction consists of 5 stages:
\subsubsection{Fetch stage}
\begin{itemize}
    \item Get instruction from memory
    \item Address is in the Program Counter Register
\end{itemize}
\subsubsection{Decode stage}
\begin{itemize}
    \item Find out the operation required
\end{itemize}
\subsubsection{Operand fetch stage}
\begin{itemize}
    \item Get operand(s) needed for operation
\end{itemize}
\subsubsection{Execute stage}
\begin{itemize}
    \item Perform required operation
\end{itemize}
\subsubsection{Register write stage}
\begin{itemize}
    \item Store the result of operation in a register
\end{itemize}
\subsection*{Example}
\begin{table}[htbp]
    \centering
    \begin{tabular}{|p{3cm}|p{4.5cm}|p{4.5cm}|p{4.5cm}|}
    \hline
    \textbf{Instruction} & \textbf{Add} (\texttt{add \$rd, \$rs, \$rt}) & \textbf{Load Word} (\texttt{lw \$rt, ofst(\$rs)}) & \textbf{Branch on Equal} (\texttt{beq \$rs, \$rt, ofst})\\
    \hline
    \textbf{Fetch} & Standard fetch & Standard fetch & Standard fetch \\
    \hline
    \textbf{Decode} & Operand fetch & Operand fetch & Operand fetch \\
    \hline
    \textbf{Operand Fetch} & 
    Read [\texttt{\$rs}] as opr1 \newline
    Read [\texttt{\$rt}] as opr2 &
    Read [\texttt{\$rs}] as opr1 \newline
    Use ofst as opr2 &
    Read [\texttt{\$rs}] as opr1 \newline
    Read [\texttt{\$rt}] as opr2 \\
    \hline
    \textbf{Execute} & Result = opr1 + opr2 & MemAddr = opr1 + opr2 & Taken = (opr1 == opr2)? \newline Target = (PC+4) + ofst $\times$ 4 \\
    \hline
    \textbf{Result Write} & Result stored in \texttt{\$rd} & Memory data stored in \texttt{\$rt} & if (Taken) \newline PC = Target \\
    \hline
    \end{tabular}
    \caption{Stages of instruction execution for add, lw, and beq}
    \end{table}
    \newpage
\subsection{Build a MIPss Processor}
\subsubsection{Fetch stage}
\subsubsection{Decode stage}
\subsubsection{ALU stage}
\subsubsection{Memory stage}
\subsubsection{Register write stage}

\end{document}
