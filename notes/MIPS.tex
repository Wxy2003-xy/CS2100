\documentclass[12pt]{article} % 12pt font size

% Packages
\usepackage[utf8]{inputenc}    % For UTF-8 character encoding
\usepackage[T1]{fontenc}       % For proper font encoding
\usepackage{lmodern}           % Improved font rendering
\usepackage{amsmath, amssymb}  % For math symbols and environments
\usepackage{graphicx}          % For including images
\usepackage{geometry}          % For adjusting page dimensions
\usepackage{hyperref}          % For clickable hyperlinks in the document
\usepackage{fancyhdr}          % For custom headers and footers
\usepackage{parskip}           % To add space between paragraphs
\usepackage{tikz}              % For drawing figures
\usepackage{booktabs}          % For improved table formatting
\usepackage{enumitem}          % For custom lists
\usepackage{caption}           % For customizing captions
\usepackage{listings}          % For code listings
\usepackage{multirow}          % For multirow tables
\usepackage{amsthm}

\lstset{
  frame=tb,
  language=C,
  aboveskip=3mm,
  belowskip=3mm,
  showstringspaces=false,
  columns=flexible,
  basicstyle={\small\ttfamily},
  numbers=none,
  numberstyle=\tiny\color{gray},
  keywordstyle=\color{blue},
  commentstyle=\color{brown},
  stringstyle=\color{orange},
  breaklines=true,
  breakatwhitespace=true,
  tabsize=3
}

% Document settings
\geometry{margin=1in} % Set all margins to 1 inch

% Header and Footer customization
\pagestyle{fancy}
\fancyhf{}
\fancyhead[L]{\leftmark} % Left header contains the section name
\fancyhead[R]{\thepage}  % Right header contains the page number

\title{MIPS}
\author{WANG Xiyu}
\date{\today}

\begin{document}

\maketitle

\tableofcontents % Optional table of contents

\section{Instruction Set Architecture}
Instruction Set Architecture includes everything programers need to know to make the machine code work correctly, and allows computer designers to talk about functions independently from the hardware that performs them. 
This abstraction allows many implementations of varyin cost and performance that can run identical software. Basically an API between software and hardware.
\section{Machine Code vs Assembly Language}
\section{Walkthrough}

\section{General Purpose Register}
\section{MIPS Assembly Language}
\subsection{General Instruction Syntax}
\subsection{Arithmetic Operation: Addition}
\subsection{Arithmetic Operation: Subtraction}
\subsection{Complex Expressions}
\subsection{Constant/Immediate Operand}
\subsection{Register Zero (\$0 or \$zero)}
\subsection{Logical Operation: Overview}
\subsection{Logical Operation: Shifting}
\subsection{Logical Operation: Bitwise AND}
\subsection{Logical Operation: Bitwise OR} 
\subsection{Logical Operation: Bitwise NOR}
\subsection{Logical Operation: Bitwise XOR}
\section{Large Constant: Case study}
\section{MIPS Basic Instruction Checklist}





% Add content for this section here.

\subsection{Subsection Title}
% Add content for this subsection here.

\subsubsection{Subsubsection Title}
% Add content for this subsubsection here.

\begin{lstlisting}
    // Add your code example here
\end{lstlisting}

\begin{table}[h]
    \centering
    \begin{tabular}{|c|c|c|}
    \hline
    Column 1 & Column 2 & Column 3 \\ \hline
    Data 1 & Data 2 & Data 3 \\ \hline
    % Add more rows as needed
    \end{tabular}
    \caption{Table caption}
\end{table}

\begin{figure}[h]
    \centering
    \includegraphics[width=0.5\textwidth]{example-image} % Replace with your image file name
    \caption{This is a sample caption.}
    \label{fig:example}
\end{figure}

\end{document}
