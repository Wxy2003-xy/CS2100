\documentclass[12pt]{article} % 12pt font size

% Packages
\usepackage[utf8]{inputenc}    % For UTF-8 character encoding
\usepackage[T1]{fontenc}       % For proper font encoding
\usepackage{lmodern}           % Improved font rendering
\usepackage{amsmath, amssymb}  % For math symbols and environments
\usepackage{graphicx}          % For including images
\usepackage{geometry}          % For adjusting page dimensions
\usepackage{hyperref}          % For clickable hyperlinks in the document
\usepackage{fancyhdr}          % For custom headers and footers
\usepackage{parskip}           % To add space between paragraphs
\usepackage{tikz}              % For drawing figures
\usepackage{booktabs}          % For improved table formatting
\usepackage{enumitem}          % For custom lists
\usepackage{caption}           % For customizing captions
\usepackage{listings}
\usepackage{multirow} % For multirow tables
\lstset{frame=tb,
  language=C,
  aboveskip=3mm,
  belowskip=3mm,
  showstringspaces=false,
  columns=flexible,
  basicstyle={\small\ttfamily},
  numbers=none,
  numberstyle=\tiny\color{gray},
  keywordstyle=\color{blue},
  commentstyle=\color{brown},
  stringstyle=\color{orange},
  breaklines=true,
  breakatwhitespace=true,
  tabsize=3
}
% Document settings
\geometry{margin=1in} % Set all margins to 1 inch

% Header and Footer customization
\pagestyle{fancy}
\fancyhf{}
\fancyhead[L]{\leftmark} % Left header contains the section name
\fancyhead[R]{\thepage}  % Right header contains the page number

\title{Introduction to C language for CS2100}
\author{WXY}
\date{\today}

\begin{document}

\maketitle

\tableofcontents % Optional table of contents

\section{von Neumann Architecture (Prinstone Architecture)}
Von Neumann Architecture describes a computer consisting of:
\begin{enumerate}
    \item Central Proessing Unit (CPU) 
    \begin{itemize}
        \item Register: special, fast but expensive memory within processor. Used to store temproray result of computation. 
        \item A control unit containing an instruction register and prgroam counter.
        \item An arithmetic/logic unit (ALU)
    \end{itemize}
    \item Memory
    \begin{itemize}
        \item Store both program and data in random-access memory (RAM)
    \end{itemize}
    \item I/O Devices
\end{enumerate}
\section{Variables in C}
A variable is identified by a name (identifier), has a data type and contains a value which could be modified. It also has an physical address in the memory.
A variable is declare with a data type. Such as int, float etc.
Variables can be initialised at declaration, otherwise would hold an unkonw value (which cannot be assumed to be 0).

Variables in C MUST be declared BEFORE usage.
Initialisation may be redundant at times. When initialsation is required but not done so warning will be given by the compiler.

\section {Data types in C}
C is a stringly typed language, which means data must has data type specified.
Type casting in C is allowed, for example:
\begin{lstlisting}
    float i = 1.5;
    int j = i;
    printf("\%d", j);
\end{lstlisting}

j will be converted to integer data type by truncating the decimal part.

\section{Program Structure in C}
A C program has 4 main parts:
\begin{itemize}
    \item Preprocessor directives: libraries used, \#define etc.
    \item Input: through stdin (using scanf), or file input.
    \item Compute: through arithmetic operations and assignment statements.
    \item Output: through stdout (using printf), or file output.
\end{itemize}
Compile C program files using command
\begin{lstlisting}
    gcc programFileExample.c -o programFileExample
\end{lstlisting}
programFileExample will be compiled to binary code in file programFileExample.
Without specifying the output file, the code will be compiled by default into a.out.
\subsection{Preprocessor Directives}
C Preprocessor consists of: 
\begin{enumerate}
    \item Inclusion of header files: libraries used;
    \item Macro expansions: defining constants;
    \item Conditional compilation
\end{enumerate}
\subsubsection{Header Files}
To use standard input and output functions such as scanf and printf, <stdio.h> must be included.
\begin{lstlisting}
    #include <stdio.h>
\end{lstlisting}
To use functions in a library, respective header file must be included.
\begin{lstlisting}
    #include <math.h>       // Library for mathematical functions   
        // must be compiled with -lm to compile
    #include <string.h>     // Library for string functions
    # include <pthreads.h>  
        // must be compiled with -pthreads
\end{lstlisting}

\subsubsection{Marco Expansion}
Compiler will do text substitution to replace all marco names in the program with the value declared.
\begin{lstlisting}
    #define PI 3.142        // use all CAP for marco
    // NOTE: No unintentional semicolon in defining constant, otherwise
    #define PI 3.142;
    // PI will be substituted by '3.142;' instead.
\end{lstlisting}

\subsection{Input/Output}
scanf(format string, input list);
printf(format string, print list);

\begin{lstlisting}
    int i;
    double j;               // declare before use
    printf("Enter a number: ");
    scanf("%d", &i);        // &: address of operator
                            // the following input will be stored at the 
                            // address of i
    printf("Enter a decimal number: ");
    scanf("%lf", &j);
    printf("You entered %d and %f", i, j);
                            // listing order of variables corresponds to 
                            // output order

    // Note: scanf uses %lf to take floating point input,
    //       while printf uses %f to print floating point output

    %5d     // an integer with width of 5, right justified
    %8.3f   // an float/double with width of 8, with 3 decimal places, right justified
\end{lstlisting}

\begin{table}[h]
    \centering
    \begin{tabular}{|c|c|c|}
    \hline
    Placeholder & Variable Type & Function Use \\ \hline
    \%c & char & printf/scanf \\ \hline
    \%d & int & printf/scanf \\ \hline
    \%f & float/double & printf \\ \hline
    \%f & float & scanf \\ \hline
    \%lf & double & scanf \\ \hline
    \%e & float/double & printf(for scientific notation) \\ \hline

    \end{tabular}
    \caption{Format specifiers}
\end{table}

\begin{table}[h]
    \centering
    \begin{tabular}{|c|c|c|}
    \hline
    Escape sequences & Meaning & Result \\ \hline
    \texttt{\textbackslash}n & New line & subsequent output will be on the next line \\ \hline
    \texttt{\textbackslash}t & Horizontal tab & Move to the next tab position within the same line \\ \hline
    \texttt{\textbackslash}" & Double quote & literal double quote character \\ \hline
    \#\# & Percent & literal charater '\%' \\ \hline

    \end{tabular}
    \caption{Escape sequences}
\end{table}

    


\begin{figure}[h]
    \centering
    \includegraphics[width=0.5\textwidth]{example-image} % Replace with your image file name
    \caption{This is a sample caption.}
    \label{fig:example}
\end{figure}

\end{document}
