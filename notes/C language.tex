\documentclass[12pt]{article} % 12pt font size

% Packages
\usepackage[utf8]{inputenc}    % For UTF-8 character encoding
\usepackage[T1]{fontenc}       % For proper font encoding
\usepackage{lmodern}           % Improved font rendering
\usepackage{amsmath, amssymb}  % For math symbols and environments
\usepackage{graphicx}          % For including images
\usepackage{geometry}          % For adjusting page dimensions
\usepackage{hyperref}          % For clickable hyperlinks in the document
\usepackage{fancyhdr}          % For custom headers and footers
\usepackage{parskip}           % To add space between paragraphs
\usepackage{tikz}              % For drawing figures
\usepackage{booktabs}          % For improved table formatting
\usepackage{enumitem}          % For custom lists
\usepackage{caption}           % For customizing captions
\usepackage{listings}
\usepackage{multirow} % For multirow tables
\lstset{frame=tb,
  language=C,
  aboveskip=3mm,
  belowskip=3mm,
  showstringspaces=false,
  columns=flexible,
  basicstyle={\small\ttfamily},
  numbers=none,
  numberstyle=\tiny\color{gray},
  keywordstyle=\color{blue},
  commentstyle=\color{brown},
  stringstyle=\color{orange},
  breaklines=true,
  breakatwhitespace=true,
  tabsize=3
}
% Document settings
\geometry{margin=1in} % Set all margins to 1 inch

% Header and Footer customization
\pagestyle{fancy}
\fancyhf{}
\fancyhead[L]{\leftmark} % Left header contains the section name
\fancyhead[R]{\thepage}  % Right header contains the page number

\title{Introduction to C language for CS2100}
\author{WXY}
\date{\today}

\begin{document}

\maketitle

\tableofcontents % Optional table of contents

\section{von Neumann Architecture (Prinstone Architecture)}
Von Neumann Architecture describes a computer consisting of:
\begin{enumerate}
    \item Central Proessing Unit (CPU) 
    \begin{itemize}
        \item Register: special, fast but expensive memory within processor. Used to store temproray result of computation. 
        \item A control unit containing an instruction register and prgroam counter.
        \item An arithmetic/logic unit (ALU)
    \end{itemize}
    \item Memory
    \begin{itemize}
        \item Store both program and data in random-access memory (RAM)
    \end{itemize}
    \item I/O Devices
\end{enumerate}
\section{Variables in C}
A variable is identified by a name (identifier), has a data type and contains a value which could be modified. It also has an physical address in the memory.
A variable is declare with a data type. Such as int, float etc.
Variables can be initialised at declaration, otherwise would hold an unkonw value (which cannot be assumed to be 0).

Variables in C MUST be declared BEFORE usage.
Initialisation may be redundant at times. When initialsation is required but not done so warning will be given by the compiler.
\subsection*{Declaration vs Definition}
\begin{itemize}
    \item Separate compilation: 
    \item \(extern\) keyword
\end{itemize}

\subsubsection*{Separate compilation and definition}
if variable is defined in both file1.c and file2.c, which will be separately compiled to file1.o and file2.o
if definition duplication cannot be resolved by checking the scopes, the linker would not be able to join the compiled files and no executable can be generated.
\section {Data types in C}
C is a stringly typed language, which means data must has data type specified.
Type casting in C is allowed, for example:
\begin{lstlisting}
    float i = 1.5;
    int j = i;
    printf("\%d", j);
\end{lstlisting}

j will be converted to integer data type by truncating the decimal part.

\section{Program Structure in C}
A C program has 4 main parts:
\begin{itemize}
    \item Preprocessor directives: libraries used, \#define etc.
    \item Input: through stdin (using scanf), or file input.
    \item Compute: through arithmetic operations and assignment statements.
    \item Output: through stdout (using printf), or file output.
\end{itemize}
Compile C program files using command
\begin{lstlisting}
    gcc programFileExample.c -o programFileExample
    cc programFileExample.c -E -o programFileExample.pp 
    // to stop at preprocessor directory
    cc programFileExample.c -S -o programFileExample.s
    // compile into assembly
\end{lstlisting}
programFileExample will be compiled to binary code in file programFileExample.
Without specifying the output file, the code will be compiled by default into a.out.
\subsection{Preprocessor Directives}
C Preprocessor consists of: 
\begin{enumerate}
    \item Inclusion of header files: libraries used;
    \item Macro expansions: defining constants;
    \item Conditional compilation
\end{enumerate}
\subsubsection{Header Files}
To use standard input and output functions such as scanf and printf, <stdio.h> must be included.
\begin{lstlisting}
    #include <stdio.h>
\end{lstlisting}
To use functions in a library, respective header file must be included.
\begin{lstlisting}
    #include <math.h>       // Library for mathematical functions   
        // must be compiled with -lm to compile
    #include <string.h>     // Library for string functions
    # include <pthreads.h>  
        // must be compiled with -pthreads
\end{lstlisting}

\subsubsection{Marco Expansion}
Compiler will do text substitution to replace all marco names in the program with the value declared.
\begin{lstlisting}
    #define PI 3.142        // use all CAP for marco
    // NOTE: No unintentional semicolon in defining constant, otherwise
    #define PI 3.142;
    // PI will be substituted by '3.142;' instead.
\end{lstlisting}

\subsection{Input/Output}
scanf(format string, input list);
printf(format string, print list);

\begin{lstlisting}
    int i;
    double j;               // declare before use
    printf("Enter a number: ");
    scanf("%d", &i);        // &: address of operator
                            // the following input will be stored at the 
                            // address of i
    printf("Enter a decimal number: ");
    scanf("%lf", &j);
    printf("You entered %d and %f", i, j);
                            // listing order of variables corresponds to 
                            // output order

    // Note: scanf uses %lf to take floating point input,
    //       while printf uses %f to print floating point output

    %5d     // an integer with width of 5, right justified
    %8.3f   // an float/double with width of 8, with 3 decimal places, right justified
\end{lstlisting}

\begin{table}[h]
    \centering
    \begin{tabular}{|c|c|c|}
    \hline
    Placeholder & Variable Type & Function Use \\ \hline
    \%c & char & printf/scanf \\ \hline
    \%d & int & printf/scanf \\ \hline
    \%f & float/double & printf \\ \hline
    \%f & float & scanf \\ \hline
    \%lf & double & scanf \\ \hline
    \%e & float/double & printf(for scientific notation) \\ \hline

    \end{tabular}
    \caption{Format specifiers}
\end{table}

\begin{table}[h]
    \centering
    \begin{tabular}{|c|c|c|}
    \hline
    Escape sequences & Meaning & Result \\ \hline
    \texttt{\textbackslash}n & New line & subsequent output will be on the next line \\ \hline
    \texttt{\textbackslash}t & Horizontal tab & Move to the next tab position within the same line \\ \hline
    \texttt{\textbackslash}" & Double quote & literal double quote character \\ \hline
    \#\# & Percent & literal charater '\%' \\ \hline

    \end{tabular}
    \caption{Escape sequences}
\end{table}

\section{Pointers}
\subsection{Pointers}
Pointers allow for the direct manipulation of memory contents, as well as bitwise operations by bit manipulation operators.
A variable contains a name, data type and an address, which is a number that indicate the physical location of the variable in the memory.
The address of a variable can be accessed by '\&';
\begin{lstlisting}
    int a = 123;
    printf("a = %d\n", a);      // a = 123
    printf("&a = %p\n", &a);    // &a = ffbff7dc
\end{lstlisting}

\subsection{Declare a Pointer}
Syntax:
\begin{lstlisting}
    //type *pointer_name
    int *a_ptr;
\end{lstlisting}
\subsection{Assign Value to a Pointer}
Since a pointer contains an address, only an address may be assigned to pointer, which is a hexadecimal number indicating the location of a variable.
\begin{lstlisting}
    int a  = 123;       // declare a variable
    int *a_ptr;         // declare a pointer points to int
    a_ptr = &a;         // assign the address of a to the pointer declared in line above
\end{lstlisting}

\subsection{Accessing Variable through Pointer}
Using dereference operator '*'
\begin{lstlisting}
    int a = 123;
    printf("a = %d\n", *a_ptr);     // 123
    printf("a = %d\n", a);          // 123
    // equivalent 
    *a_ptr = 456;                   // rewrite variable ptr points To
    printf("a = %d\n", a);          // 456, value modified through referencing from the pointer
\end{lstlisting}
\[*a_ptr \cong a\]

Example:
\begin{lstlisting}
    int i = 10, j = 20;
    int *p;
    p = &i;
    printf("i = %d\n", *p);     // i = 10
    *p = *p + 2;
    printf("i = %d\n", *p);     // i = 12
    
    p = &j;
    printf("j = %d\n", *p);     // j = 20
    *p = i;
    printf("j = %d\n", *p);     // j = 12
\end{lstlisting}

Example:
\begin{lstlisting}
    #include<stdio.h>

    int main() {
        double a, *b;            // a is a double; *b is a pointer to a double variable
        b = &a;
        *b = 12.34;
        printf("a = %f\n", a);   // a = 12.34
        printf("a = %f\n", *b);  // a = 12.34

        printf("a = %f\n", b);   // Compilation warning, b is a pointer
        printf("a = %f\n", a);   // Compilation error, 
                                 //cannot dereference from a double (segmentation fault)
    }
\end{lstlisting}

\subsection{Tracing Pointers}
\begin{lstlisting}
    int a = 8, b = 15, c = 23;
    int *p1, *p2, *p3;
    p1 = &b;        // make p1 points to b, which is 15;
    p2 = &c;        // make p2 points to c, which is 23;
    p3 = p2;        // make p3 points to the same thing as p2, which is c;
    printf("1: %d %d %d\n", *p1, *p2, *p3);     // 1: 15 23 23;
    
    *p1 *= a;       // multiply the variable pointed by p1 by a; b => b * a = 90;
    while (*p2 > 0) {       // p2 points to c, 
                            // *p2 is dereferenced to be the value of c: 23;
        *p2 -= a;           // in each loop iteration, decrement the c by a;
        (*p1)++;            // and increment the value that p1 points to by 1;
    }
    printf("2: %d %d %d\n", *p1, *p2, *p3);     // 2: 123 -1 -1
    printf("3: %d %d %d\n", a, b, c);           // 3: 8 123 -1

\end{lstlisting}
\subsection{Incrementing a Pointer}
The value of pointers can also be directly manipulated to locate neigbhouring memory locations.
\begin{lstlisting}
    int a; float b; char c; double d;
    int *ap; float *bp; char *cp; double *dp;
    // int takes 4 bytes
    // float takes 4 bytes
    // char takes 1 byte;
    // double takes 8 bytes;
    ap = &a; bp = &b; cp = &c; dp = &d;
    printf("%p %p %p %p\n", ap, bp, cp, dp); 
    // ffbff0a4 ffbff0a0 ffbff09f ffbff090
    ap++; bp++; cp++; dp++;
    printf("%p %p %p %p\n", ap, bp, cp, dp);
    // ffbff0a8 ffbff0a4 ffbff0a0 ffbff098  each pointer increment by size of respective type.
    ap += 3;
    printf("%p\n", ap);
    // ffbff0b4     incremented by 12 bytes which is the size of 3 int
\end{lstlisting}
\subsection{Pointer to Pointer}
When declaring a pointer to a type, e.g. int *p, the pointer will also be stored in a physical address. Of which another pointer can point to, e.g. int **q = \&p;
\begin{lstlisting}
    int main(){
    int *p; // Pointer to an int
    int **q; // Pointer to a pointer to an int.
    int x=5, y = 6;
    printf("%p, %p, %d, %d\n", p, q, x, y);     
    printf("%p, %p, %d, %d\n", &x, &y, x, y);
    
    p = &x;
    q = &p;
    p++;
    (*p)--;  
    (**q)++;   

    printf("%p, %p, %d, %d\n", p, q, x, y);
    return 0;
    }
\end{lstlisting}



\end{document}
